\documentclass[11pt]{article}
\usepackage[sexy, fancy]{evan}
\usepackage{hyperref}

\renewcommand{\arraystretch}{2}

\begin{document}

\title{Combinatorics - Week 5}
\maketitle

\section*{Introduction}
We're going to move on from doing Algebra now; there's still some topics left in Algebra, but for the sake of time we'll switch to Combinatorics for a little while. Combinatorics, as defined just refers to the mathematics of couting objects, though
this definition leaves out many applications of the subject. 

\subsection*{Notation}
Combinatorics introduces some more notation. We need to define some special sets and expand summation notation a bit. The set $[n]$ is shorthand for $\{ 1, 2, \cdots n \}$; this set is very commonly seen for reasons we'll see in a bit. We'll say 2 sets have a one to one correspondence, or a bijection,  if we can pair every element in one set with an element of another. This means that there exists a function which maps every element in one set to another element in another. In our case, 
what this means in practice is that one set is \textit{analagous} to another, so the set $[n]$ is equivalent to $n$ people in a line, a set of $n$ apples, a set of $n$ balls, or any set of $n$ random object, because we're able to assign each number to a unique object. We can also extend summations to summing over sets. If we have a set $S$, the notation $\sum_{s \in S} f(s)$ means that we apply the function $f$ to every element in $S$ and then sum the outputs. The case $\sum_{i=1}^n f(i)$ would be equivalent to $\sum_{i \in [n]} f(i)$.
\section{Theory}

\subsection{Permutations}

\subsection{Combinations}

\subsection{Binomial Theorem}


\section{Methods}

\subsection{Constructive}

\subsection{Complementary}

\subsection{Casework}

\section{Further Reading}

\end{document}
