\documentclass[11pt]{article}
\usepackage[sexy, fancy, mdthm]{evan}
\usepackage{hyperref}
\usepackage{polynom}

\begin{document}

\title{Geometry Survival Guide}
\maketitle

\section*{What's the Purpose of This?}
\par So, why am I making this long document? Well, I likely won't be able to cover
much geometry in the club (because there's too much to learn and drawing diagrams takes time), so I decided to compile a general list of all the theorems and techniques
for competition math you would probably need to know for geometry. Use this as a reference to see what you need to learn, all the exercises
here are not at all required but are good for improving your understanding. Also, if you're wondering what compelled me to make such a large document, it wasn't because im so altruistic (although I am),  it was partly so I could do some
review for AMC and AIME this year and also to learn Asymptote (makes geometry figures).

\tableofcontents
\section{Reading}
I'll give the standard recommendations first; AOPS Vol. 1 \& 2 have several chapters dedicated to geometry and are good sources for learing most of the material. \textit{A Beautiful Journey Through Olympiad Geometry}, while this book is proof based, it covers geometry from the beginning rather well
and the early chapters are a good read even if you only plan on doing computational problems. 
\subsection*{Handouts}
A list of handouts that I found useful myself when learning the material or reviewing.
\begin{itemize}
    \item
\end{itemize}
\section{Notation}
\begin{itemize}
    \item $\triangle ABC$ denotes a triangle with vertices $A$, $B$, and $C$ 
    \item $\overline{AB}$ denotes a line segment with vertices $A$ and $B$
    \item $\overline{AB} \parallel \overline{CD}$ denotes $\overline{AB}$ is parallel with $\overline{CD}$
    \item $\overline{AB} \cap  \overline{CD}$ denotes the intersection of line segments $\overline{AB}$ and $\overline{CD}$
    \item $[ABC]$ denotes the area of $\triangle ABC$.
\end{itemize}
\section{Necessary}
This document has a lot of information, but if there's only a few things to take away from it, it would be this first section. This first section should outline you the tools
to at least approach geometry problems. A brief description of the content in this section is that most of the content in this section should be covered in Honors Geometry with a few extra topics that are touched on. 
\subsection{Triangles}
\subsubsection*{Properties}
\begin{itemize}
    \item The angles in a triangle add to $180^{\circ}$.
    \item (Triangle Ineqality) if $BC=a$, $AC=b$, $AB=c$ and $a < b < c$, then $a+b > c$. 
\end{itemize}
\subsubsection*{Congruent \& Similar}
\begin{itemize}
    \item 2 triangles are congruent ( $\cong$), if they are the same triangle
    \item 3 conditions for congruency:
     \begin{itemize}
            \item (SSS) - All sides are the same
            \item (ASA) - 2 angles and the side adjacent to both angles are equal
            \item (SAS) 2 sides and the angle inbetween are equal
            \item (AAS) - 2 angles and the next side match
            \item \textbf{WARNING: } ASS, 2 sides and an angle not inbetween the two are equal, does \textbf{NOT} imply congruency or similarity
        \end{itemize}
    \item 2 triangles are similar ($\sim$) if one triangle is a scaled version of the other. 
    \item 1 conditions for similarity:
     \begin{itemize}
            \item (AA) - 2 angles in both triangles are equal (third can be calculated)
        \end{itemize}
    \item if $\triangle ABC \sim \triangle XYZ$, and the scale ratio is $r$, then the scale ratio of the areas, $[ABC] : [XYZ]$, is $r^2$.
\end{itemize}
\subsubsection*{Area}
\begin{itemize}
    \item $[ABC]=\frac{1}{2}bh$ where $b$ is the base and $h$ is a altitude, or height.
    \item $[ABC]=\frac{1}{2}ab sin \theta$ where $a$ and $b$ are side lengths of the triangle and $\theta$ is the angle inbetween.
\end{itemize}
\subsection{Circles}

\subsubsection*{Definition}
\subsubsection*{Angles}
\subsubsection*{Area}
\subsection{Angle Chasing}
\subsubsection*{Do It}
\subsection{Quadrilaterals} 
\subsubsection*{Properties}
\subsubsection*{Types of Quadrilaterals}
\subsection{Polygons}
\subsubsection*{Properties}
\subsubsection*{Area}
\subsection{Analytic}
\subsubsection*{Coordinates}
\subsection{Basic Trig}
\subsubsection*{SOH-CAH-TOA}
\subsubsection*{Properties}
\subsubsection*{Using it}
\subsection{3D Geometry}
\subsubsection*{Shapes}
\subsubsection*{Formulas}
\section{AMC 10}
\subsection{Area Formulas}
\subsubsection*{Special Ones}
\subsection{More Triangles}
\subsubsection*{Theorems}
\subsection{Cyclic Quadrilaterals}
\subsubsection*{Definition}
\subsubsection*{Theorems}
\subsection{Power of a Point}
\subsubsection*{Definition \& Formulas}
\subsection{Trigonometry}
\subsubsection*{More Formulas}
\subsubsection*{Properties}
\subsection{Analytic Geometry}
\subsubsection*{Combining with synthetic}
\subsubsection*{Special lines}
\section{AMC 12}
\subsection{Even more Triangles}
\subsubsection*{Properties}
\subsection{Colinearity}
\subsubsection*{Theorems}
\subsection{Complex Numbers}
\subsubsection*{Applications to Geometry}
\subsection{Vectors}
\subsubsection*{Definition \& Formulas}
\subsubsection*{Applications to Geometry}
\subsection{3D Geometry}
\subsubsection*{Definitions}
\subsubsection*{Theorems}
\subsection{Mass Points}
\section{AIME}
\subsection{Barycentric Coordinates}
\subsection{Projective Geometry}
\subsection{Olympiad Geometry}

\end{document}